% -----
% ARQUIVO: capitulo-04.tex
% VERSÃO: 1.1
% DATA: Janeiro de 2016
%
% CAPÍTULO DE METODOLOGIA DA PROPOSTA
%
% NÃO MEXA NAS SEÇÕES, SOMENTE EDITE O CONTEÚDO.
% -----

\chapter{A Proposta}
% #TXT_INTROPROPOSTA
\lipsum[1]

\section{Questões de Pesquisa}
% #TXT_QUESTÕES
\textcolor{RedOrange}{Uma questão de pesquisa é a declaração de uma indagação específica que o pesquisador deseja responder para abordar o problema de pesquisa. A(s) questão(ões) de pesquisa resultam da revisão da literatura feita no item anteior e orienta(m) os tipos de dados a serem coletados, o tipo de estudo a ser desenvolvido e os resultados esperados. Escrever tantas questões quantas necessárias para a pesquisa.}

Ut imperdiet, enim sed gravida sollicitudin, felis odio placerat quam, ac pulvinar elit purus eget enim. Nunc vitae tortor. Proin tempus nibh sit amet nisl. Vivamus quis tortor vitae risus porta vehicula. Fusce mauris.

\begin{qpesq}
Lorem ipsum dolor sit amet, consectetur adipiscing elit, sed do eiusmod tempor incididunt ut labore et dolore magna aliqua?
\end{qpesq}

Pellentesque a nulla. Cum sociis natoque penatibus et magnis dis parturient montes, nascetur ridiculus mus. Aliquam tincidunt urna. Nulla ullamcorper vestibulum turpis. Pellentesque cursus luctus mauris.

\begin{qpesq}
Lorem ipsum dolor sit amet, consectetur adipiscing elit, sed do eiusmod tempor incididunt ut labore et dolore magna aliqua?
\end{qpesq}

Morbi blandit ligula feugiat magna. Nunc eleifend consequat lorem. Sed lacinia nulla vitae enim. Pellentesque tincidunt purus vel magna. Integer non enim. Praesent euismod nunc eu purus. Donec bibendum quam in tellus. Nullam cursus pulvinar lectus.


\section{Objetivos}
% #TXT_OBJETIVO
\textcolor{RedOrange}{Iniciar com a declaração da hipótese do trabalho de pesquisa, se necessário.}

\textcolor{RedOrange}{Apresentar o objetivo geral (responder a pergunta ``onde você quer chegar com este trabalho?'') e os objetivos específicos (responder a pergunta ``o que deve ser gerado após a conclusão do trabalho?''). Os objetivos específicos devem ser apresentados em tópicos e na ordem em que serão realizados.}

Aliquam pellentesque, augue quis sagittis posuere, turpis lacus congue quam, in hendrerit risus eros eget felis. Maecenas eget erat in sapien mattis porttitor. Vestibulum porttitor. Nulla facilisi.


\section{Contribuições Esperadas}
% #TXT_CONTRIBUIÇÕES
\textcolor{RedOrange}{Levando-se em conta a revisão da literatura feita no item anterior devem ser ressaltadas as contribuições para a área, como possibilidade para uma dissertação de mestrado. Neste item devem ser listadas as contribuições esperadas e o diferencial em relação aos trabalhos similares já analisados.}

As contribuições esperadas para este trabalho são:

\begin{enumerate}[label=(\roman*)]
\item Neque porro quisquam est, qui dolorem ipsum quia dolor sit amet, consectetur, adipisci velit, sed quia non numquam eius modi tempora incidunt ut labore et dolore magnam aliquam quaerat voluptatem.

\item At vero eos et accusamus et iusto odio dignissimos ducimus qui blanditiis praesentium voluptatum deleniti atque corrupti quos dolores et quas molestias excepturi sint occaecati cupiditate non provident, similique sunt in culpa qui officia deserunt mollitia animi, id est laborum et dolorum fuga.
\end{enumerate}


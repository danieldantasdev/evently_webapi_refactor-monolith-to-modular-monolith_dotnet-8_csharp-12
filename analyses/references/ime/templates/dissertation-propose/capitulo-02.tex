% -----
% ARQUIVO: capitulo-02.tex
% VERSÃO: 1.1
% DATA: Janeiro de 2016
%
% CAPÍTULO DE INTRODUÇÃO DA PROPOSTA
%
% NÃO MEXA NAS SEÇÕES, SOMENTE EDITE O CONTEÚDO.
% -----

\chapter{Introdu\c{c}\~{a}o}
% #TXT_INTRODUCAO
\textcolor{RedOrange}{Na introdução devem ser expostos o tema do trabalho, a relevância do tema, uma breve citação sobre os trabalhos relacionados ao tema, o problema a ser abordado (que deve resultar de uma breve descrição das limitações dos trabalhos relacionados apresentados) e o objetivo geral do trabalho (responder a pergunta ``onde você quer chegar com este trabalho?'').}

Lorem ipsum dolor sit amet, consectetuer adipiscing elit. Ut purus elit, vestibulum ut, placerat ac, adipiscing vitae, felis \citep{Salles2014}. Curabitur dictum gravida mauris. Nam arcu libero, nonummy eget, consectetuer id, vulputate a, magna. Donec vehicula augue eu neque. Pellentesque habitant morbi tristique senectus et netus et malesuada fames ac turpis egestas. Mauris ut leo. Cras viverra metus rhoncus sem. Nulla et lectus vestibulum urna fringilla ultrices. Phasellus eu tellus sit amet tortor gravida placerat. Integer sapien est, iaculis in, pretium quis, viverra ac, nunc.

Ut imperdiet, enim sed gravida sollicitudin, felis odio placerat quam, ac pulvinar elit purus eget enim \citep{Gubitoso1992}. Nunc vitae tortor. Proin tempus nibh sit amet nisl. Vivamus quis tortor vitae risus porta vehicula. Fusce mauris. Vestibulum luctus nibh at lectus. Sed bibendum, nulla a faucibus semper, leo velit ultricies tellus, ac venenatis arcu wisi vel nisl \citep{icse2015}. Aliquam pellentesque, augue quis sagittis posuere, turpis lacus congue quam, in hendrerit risus eros eget felis. Maecenas eget erat in sapien mattis porttitor. Vestibulum porttitor. Nulla facilisi. Sed a turpis eu lacus commodo facilisis. Morbi fringilla, wisi in dignissim interdum, justo lectus sagittis dui, et vehicula libero dui cursus dui.

\section{Justificativa}
% #TXT_MOTIVACAO
\textcolor{RedOrange}{Apresentar os motivos que levaram \`{a} pesquisa e a sua importância enquadrando-a em uma linha de pesquisa do Programa.}

Segundo \citet{Goldschmidt2005}, cras nec ante. Pellentesque a nulla. Cum sociis natoque penatibus et magnis dis parturient montes, nascetur ridiculus mus. Aliquam tincidunt urna \citep{Rakocevic2014}. Nulla ullamcorper vestibulum turpis. Pellentesque cursus luctus mauris. Nulla malesuada porttitor diam. Donec felis erat, congue non, volutpat at, tincidunt tristique, libero. Vivamus viverra fermentum felis. Donec nonummy pellentesque ante. Phasellus adipiscing semper elit. Proin fermentum massa ac quam \citep{Lara2014}. Sed diam turpis, molestie vitae, placerat a, molestie nec, leo. Maecenas lacinia. Nam ipsum ligula, eleifend at, accumsan nec, suscipit a, ipsum.

Morbi blandit ligula feugiat magna. De acordo com  \citet{Soares2013} nunc eleifend consequat lorem. Sed lacinia nulla vitae enim. Pellentesque tincidunt purus vel magna. Integer non enim. Praesent euismod nunc eu purus. Donec bibendum quam in tellus \citep{Yoko2003}. Nullam cursus pulvinar lectus. Donec et mi. Nam vulputate metus eu enim. Vestibulum pellentesque felis eu massa. Quisque ullamcorper placerat ipsum. Cras nibh. Morbi vel justo vitae lacus tincidunt ultrices. Lorem ipsum dolor sit amet, consectetuer adipiscing elit. In hac habitasse platea dictumst. Em \citet{Dias2013} integer tempus convallis augue. Etiam facilisis. Nunc elementum fermentum wisi. Aenean placerat.

